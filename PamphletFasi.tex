\documentclass{article}
\usepackage{xepersian}
\settextfont[Scale=1.0]{Vazirmatn}         % متن فارسی
\setlatintextfont[Scale=1.0]{Times New Roman} % متن انگلیسی
\setdigitfont{Vazirmatn}   
\usepackage{listings}
\usepackage{xcolor}
\usepackage[a4paper, margin=2.5cm]{geometry}
\lstset{
  basicstyle=\ttfamily\small,
  keywordstyle=\color{blue},
  stringstyle=\color{red},
  commentstyle=\color{gray},
  numbers=left,
  numberstyle=\tiny\color{gray},
  breaklines=true,
  frame=single
}

\usepackage{tikz}
\usetikzlibrary{calc}
\usepackage{eso-pic}
\begin{document}
\AddToShipoutPictureBG{%
  \begin{tikzpicture}[remember picture,overlay]
    % ضخامت خط کادر را تنظیم کن
    \draw[line width=1pt]
      % فاصله کادر از لبه برگ (قابل تغییر)
      ($(current page.north west) + (1.2cm,-1.2cm)$)
        rectangle
      ($(current page.south east) + (-1.2cm,1.2cm)$);
  \end{tikzpicture}
}

دوره آموزشی برنامه متمتیکا
\par
دوره سایت فرادرس
\par
جلسه اول
\par
حذف متغیر در برنامه (هر متغیری قبل از این دستور پاک می‌شود.)
\par
\begin{LTR}
\begin{lstlisting}
Clear["Global`*"]
\end{lstlisting}
\end{LTR}
\par
تنها متغیر a حذف می‌شود.
\begin{LTR}
\begin{lstlisting}
Clear[a]
\end{lstlisting}
\end{LTR}
\par
غیرفعال کردن یک دستور:
\par
\LR{edit منوی} $\leftarrow$  Section Un/comment
\par
کلید میانبر :‌ / + Alt
\par
نشان دادن هر متغیری (چه مقدار دهی شده و چه مقدار دهی نشده)
\par
\begin{LTR}
\begin{lstlisting}
Clear["Global`*"]
Clear["Name`*"]
\end{lstlisting}
\end{LTR}
\par
* همیشه باید در اول هر برنامه از دو کد Remove  و Clear استفاده کنیم تا در محاسبات مشکلی پیش نیاید.
\par
* در متمتیکا حرف اول همه دستورها باید بزرگ شود.
\par
* روش اجرا دستورهای در یک سل :‌ Enter + Shift
\par
اجرای دستورهای همه سل‌ها به صورت یکجا:
$Evaluation \rightarrow Evaluate Notebook$
\par
\par
دستور برای مقداردهی یک متغیر:
\par
\begin{LTR}
\begin{lstlisting}
Variable_name = Value
\end{lstlisting}
\end{LTR}
\par
برای مثال در کد زیر مقدار 2 را درون متغیری با نام a قرار می‌دهیم:
\begin{LTR}
\begin{lstlisting}
a = 2
\end{lstlisting}
\end{LTR}
\par
مشکلی که این نوع مقداردهی دارد این است که بعد از اجرا سلولی که این کد درون آن است، این مقدار درخروجی چاپ می‌شود. برای اینکه هم مقدار موردنظر در متغیر موردنظر ذخیره شود و هم اینکه در خروحی نمایش داده نشود. دو روش وجود دارد. روش اول:
\par
\begin{LTR}
\begin{lstlisting}
a = 2;
\end{lstlisting}
\end{LTR}
\par
روش دیگری که موجود است:
\begin{LTR}
\begin{lstlisting}
a := 2
\end{lstlisting}
\end{LTR}
\par
تقاوت این دو دستور باهم این است که در روش اول مقدار حساب و ذخیره می‌شوند . ولی در روش دوم مقدار حساب نمی‌شود تا زمانی که از آن در جایی استفاده شود
\par
روش متوقف کردن برنامه در حال اجرا:
\par
\LR{Evaluation} ⇒ \LR{Quit kernel} ⇒ \LR{Local}
\par
جواب سوال پرسیده شده: Quit
\par
* هر دستور در متمتیکا باید در براکت ([ ]) نوشته شود:
\par
\begin{LTR}
\begin{lstlisting}
Sin[x] , Clear[c]
\end{lstlisting}
\end{LTR}
\par
دستورهای پایه ریاضی
\par
برای این بخش دو متغیر از قبل تعریف می‌کنیم:

\par
\begin{LTR}
\begin{lstlisting}
a = 5
b = 8
\end{lstlisting}
\end{LTR}
\par
دستور جمع:
\par
\begin{LTR}
\begin{lstlisting}
a + b        Out : 13
\end{lstlisting}
\end{LTR}
\par
دستور تفریق:
\par
\begin{LTR}
\begin{lstlisting}
a - b        Out : -3
\end{lstlisting}
\end{LTR}
\par
دستور ضرب:
\par
\begin{LTR}
\begin{lstlisting}
a * b        Out : 40
\end{lstlisting}
\end{LTR}
\par
دستور تقسیم (روش اول):
\par
\begin{LTR}
\begin{lstlisting}
a / b        Out : 5 / 8
\end{lstlisting}
\end{LTR}
\par
دستور تقسیم (روش دوم):
\par
\begin{LTR}
\begin{lstlisting}
a/b        Out : 5 / 8
\end{lstlisting}
\end{LTR}
\par
نحوه خروجی گرفتن با دقت موردنظر:
\begin{LTR}
\begin{lstlisting}
N[expression]                         e.g: N[1/3]        Out: 0.333
N[expression , accuracy]       e.g: N[1/3 , 10]        Out: 0.3333333333
\end{lstlisting}
\end{LTR}
\par
\par
برای نوشتن این روش پس از نوشتن صورت تقسیم (در این مثال a) / + Ctrl را می‌زنیم. 
\par
روش تایپ کردن فرمول‌های خاص:
\par
$Palettes \rightarrow Basic Math Assistant$
\par
استفاده از حروف یونانی و اسمبل‌ها در این قسمت نیر موجود است. 
\par
هز گزینه کلید میانبر مخصوص خود را دارد. برای آشنایی با کلید میانبر هر گزینه، روی گزینه کمی نگه دارید.
\par
* خطای 1042 : ایجاد اشکال در دستور وارد شده، مثال:
\par
\begin{LTR}
\begin{lstlisting}
W = W + 1   ,  W = W1 + 1
\end{lstlisting}
\end{LTR}
\par
روش دیگر برای نوشتن این عبارت: در این حالت کنترلی در تعداد اعشار نداریم.
\par
\begin{LTR}
\begin{lstlisting}
1 / 3 // N
\end{lstlisting}
\end{LTR}
\par
\par
\par
\par
جلسه دوم : معرفی توابع
\par
تابع رادیکال:
\par
\begin{LTR}
\begin{lstlisting}
Sqrt[4]                Out: 2
Sqrt[x**2]             Out: √(x^2)
\end{lstlisting}
\end{LTR}
\par
تابع ساده سازی (Simplify) :
\par
\begin{LTR}
\begin{lstlisting}
Simplify[%, x > 0]                         Out: +x
Simplify[%, x < 0]                         Out: -x
\end{lstlisting}
\end{LTR}
\par
در این مثال نماینده آخرین خروجی است.
\par
تابع نمایی:
\par
\begin{LTR}
\begin{lstlisting}
Exp[x]                            Out: e**x      e = 2.7
Exp[2]                            Out: e**2
Exp[2] // N                    Out: 7.38906
\end{lstlisting}
\end{LTR}
\par
تابع جزء صحیح:
\par
\begin{LTR}
\begin{lstlisting}
Floor[2.3]             Out: 2
\end{lstlisting}
\end{LTR}
\par
تابع قدرمطلق:
\par
\begin{LTR}
\begin{lstlisting}
Abs[-2]                         Out: 2
Abs[-x]                         Out: Abs[x]
\end{lstlisting}
\end{LTR}
\par
* در مثال دوم این کد، خروجی از تابع بیرون در نمی‌آید چون x می‌تواند منفی و چیزهای دیگر باشد.
\par
تابع علامت:
\par
\begin{LTR}
\begin{lstlisting}
Sign[{-2, 0, 3}]                         Out: {-1, 0, 1}
\end{lstlisting}
\end{LTR}
\par
تابع فاکتوریل (روش اول) :
\par
\begin{LTR}
\begin{lstlisting}
x = 5
Facorial[x]                       Out: 120
\end{lstlisting}
\end{LTR}
\par
تابع فاکتوریل (روش دوم) :
\par
\begin{LTR}
\begin{lstlisting}
x!                Out: 120
\end{lstlisting}
\end{LTR}
\par
تابع لگاریتم:
\par
\begin{LTR}
\begin{lstlisting}
Log[x]                Out: Log[x]
\end{lstlisting}
\end{LTR}
\par
زمانی که برای الگوریتم پایه تغریف نکنیم، نرم افزار به صورت خودکار پایه را عدد نپر ( ln x یا E ) قرار می‌دهد.
\par
\begin{LTR}
\begin{lstlisting}
Log[E]               Out: 1
\end{lstlisting}
\end{LTR}
\par
تغریف پایه الگوریتم:
\par
\begin{LTR}
\begin{lstlisting}
Log[a , x]                  a is The base of the logarithm
Log[10 , 1000]                Out: 3
\end{lstlisting}
\end{LTR}
\par
توابع مثلثاتی:
\par
\begin{LTR}
\begin{lstlisting}
Sin[x]                e,g: Sin[Pi/3]                Out: sqrt[3] / 2
Cos[x]
Tan[x]
Cot[x]
ArcSin[x]
ArcTan[x]
\end{lstlisting}
\end{LTR}
\par
به صورت دیفالت مقدار محاسبه این توابع به صورت رادیان است، اگر بخواهیم نرم افزار حاص درجه موردنظر را حساب کند، این گونه عمل می‌کنیم:
\par
\begin{LTR}
\begin{lstlisting}
Sin[30 Degree]                Out: sqrt[3] / 2
Sin[30 Degree] // N           Out: 0.5
\end{lstlisting}
\end{LTR}
\par
\par
توابع هیپربولیک
\par
\begin{LTR}
\begin{lstlisting}
Sinh[x]
Cosh[x]
Coth[x]
\end{lstlisting}
\end{LTR}
\par
تابع تولید اعداد تصادفی
\par
برای اعداد حقیقی:

\par
\begin{LTR}
\begin{lstlisting}
RandomReal[{-2, 2}, 5]     Out: {1.31496, 0.0146972, -1.486, -1.93584, 0.638742}
RandomReal[{-2, 2}, {2, 5}] Out:{{1.86401, -1.47587, 0.0531725, 1.73636, 0.243321}, 
{1.50354, 0.924516, 1.60522, 0.870081, -1.22427}}
\end{lstlisting}
\end{LTR}
\par
در مثال اول پنج عدد تصادفی در بازه {2 ,  2-} انتخاب می‌شوند.
\par
در مثال دوم دو دسته پنج تایی عدد تصادفی در بازه {2 ,  2-} انتخاب می‌شوند.
\par
برای اعداد صحیح:
\par
\begin{LTR}
\begin{lstlisting}
RandomInteger[{-2, 2}, 5]      Out: {-2, 0, 1, 2, -2}
\end{lstlisting}
\end{LTR}
\par
پنج عدد صحیح تصادفی در بازه {2 ,  2-} انتخاب می‌شوند.
\par
تبدیل عدد به عامل های اول:
\par
\begin{LTR}
\begin{lstlisting}
FactorInteger[10]     Out: {{2, 1}, {5, 1}}      10 = 2^1 * 5^1
\end{lstlisting}
\end{LTR}
\par
به توان رساندن:
\par
\begin{LTR}
\begin{lstlisting}
Superscript[2, 3]        Out: Superscript[2,3]
Superscript @@ {2, 3}     Out: Superscript[2,3]
Superscript @@@ {{2, 3}, {6, 5}}        Out: {Superscript[2,3], Superscript[6,5]}
Superscript @@ {{2, 3}, {4, 5}}         Out: Superscript[{2, 3},{4, 5}]
\end{lstlisting}
\end{LTR}
\par

\par
\begin{LTR}
\begin{lstlisting}
CenterDot[x, y]           Out: x\[CenterDot]y
CenterDot @@ (Superscript @@@ (FactorInteger[15]))  Out: Superscript[3,1]\[CenterDot]Superscript[5,1]
\end{lstlisting}
\end{LTR}
\par

کوچک‌ترین مضرب مشترک
\par
\begin{LTR}
\begin{lstlisting}
LCM[5, 6]
\end{lstlisting}
\end{LTR}
\par
بزرگترین تقسیم الیه مشترک
\par
\begin{LTR}
\begin{lstlisting}
GCD[15, 9]
\end{lstlisting}
\end{LTR}
\par
باقی‌مانده
\par
\begin{LTR}
\begin{lstlisting}
Mod[15 , 2]
\end{lstlisting}
\end{LTR}
\par
خارج قسمت
\par
\begin{LTR}
\begin{lstlisting}
Quotient[15, 3]
\end{lstlisting}
\end{LTR}
\par
ترکیب
\par
\begin{LTR}
\begin{lstlisting}
Binomial[15, 3]
\end{lstlisting}
\end{LTR}
\par
دلتار دیراک
\par
\begin{LTR}
\begin{lstlisting}
DiracDelta[x]
DiracDelta[x] // TraditionalForm   Out: \[Delta](x)
\end{lstlisting}
\end{LTR}
\par
دلتا کراینیکر
\par
\begin{LTR}
\begin{lstlisting}
DiracDelta[m, n] // TraditionalForm
\end{lstlisting}
\end{LTR}
\par
اعداد مختلط:
\par
\begin{LTR}
\begin{lstlisting}
z = x + I y    Out: x + I y
Re[z]          Out: -Im[y] + Re[x]
\end{lstlisting}
\end{LTR}
\par
برای تعریف کردن قسمت موهومی و واقعی اعداد مختلط:
\par
\begin{LTR}
\begin{lstlisting}
Refine[Re[z], Element[{x, y}, Reals]]        Out: x               // Real Part
Refine[Im[z], Element[{x, y}, Reals]]        Out: y              // Imaginary Part
\end{lstlisting}
\end{LTR}
\par
مزدوج گیری (مختلط) :
\par
\begin{LTR}
\begin{lstlisting}
Conjugate[z]        Out: Conjugate[x] - I Conjugate[y]
\end{lstlisting}
\end{LTR}
\par
ریفاین کردن

\par
\begin{LTR}
\begin{lstlisting}
Refine[Conjugate[z], Element[{x, y}, Reals]]        Out: x - I y
\end{lstlisting}
\end{LTR}
\par
تعریف تابع دلخواه:

\par
\begin{LTR}
\begin{lstlisting}
F[x_] = x^2 + 1   Out: 1 + x^2
F[3]              Out: 10
Map[F, {2, 3}]    Out: {5, 10}
F /@ {2, 3}       Out: {5, 10}
\end{lstlisting}
\end{LTR}
\par
برای تعریف توابع چند متغیره:

\par
\begin{LTR}
\begin{lstlisting}
G[x_, y_] = x + y + 2     Out: 2 + x + y
G[{2, 3}, {5, 6}]         Out: {9, 11}
\end{lstlisting}
\end{LTR}
\par
\par
\par
\par
\par
جلسه سوم : محسابات جبری و مثلثاتی، سری‌ها
\end{document}