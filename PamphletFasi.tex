\documentclass{article}
\usepackage{xepersian}
\settextfont[Scale=1.0]{Vazirmatn}
\setlatintextfont[Scale=1.0]{Times New Roman}
\setdigitfont{Vazirmatn}   
\usepackage{listings}
\usepackage{xcolor}
\usepackage[a4paper, margin=2.5cm]{geometry}
\usepackage{graphicx}
\usepackage{amsmath}
\lstset{
  basicstyle=\ttfamily\small,
  keywordstyle=\color{blue},
  stringstyle=\color{red},
  commentstyle=\color{gray},
  numbers=left,
  numberstyle=\tiny\color{gray},
  breaklines=true,
  frame=single
}

\usepackage{tikz}
\usetikzlibrary{calc}
\usepackage{eso-pic}
\begin{document}
\AddToShipoutPictureBG{%
  \begin{tikzpicture}[remember picture,overlay]
    % ضخامت خط کادر را تنظیم کن
    \draw[line width=1pt]
      % فاصله کادر از لبه برگ (قابل تغییر)
      ($(current page.north west) + (1.2cm,-1.2cm)$)
        rectangle
      ($(current page.south east) + (-1.2cm,1.2cm)$);
  \end{tikzpicture}
}

دوره آموزشی برنامه متمتیکا
\par
دوره سایت فرادرس
\par
جلسه اول
\par
حذف متغیر در برنامه (هر متغیری قبل از این دستور پاک می‌شود.)
\par
\begin{LTR}
\begin{lstlisting}
Clear["Global`*"]
\end{lstlisting}
\end{LTR}
\par
تنها متغیر a حذف می‌شود.
\begin{LTR}
\begin{lstlisting}
Clear[a]
\end{lstlisting}
\end{LTR}
\par
غیرفعال کردن یک دستور:
\par
\LR{edit منوی} $\leftarrow$  Section Un/comment
\par
کلید میانبر :‌ / + Alt
\par
نشان دادن هر متغیری (چه مقدار دهی شده و چه مقدار دهی نشده)
\par
\begin{LTR}
\begin{lstlisting}
Clear["Name`*"]
\end{lstlisting}
\end{LTR}
\par
* همیشه باید در اول هر برنامه از دو کد Remove  و Clear استفاده کنیم تا در محاسبات مشکلی پیش نیاید.
\par
* در متمتیکا حرف اول همه دستورها باید بزرگ شود.
\par
* روش اجرا دستورهای در یک سل :‌ Enter + Shift
\par
اجرای دستورهای همه سل‌ها به صورت یکجا:
$Evaluation \rightarrow Evaluate Notebook$
\par
\par
دستور برای مقداردهی یک متغیر:
\par
\begin{LTR}
\begin{lstlisting}
Variable_name = Value
\end{lstlisting}
\end{LTR}
\par
برای مثال در کد زیر مقدار 2 را درون متغیری با نام a قرار می‌دهیم:
\begin{LTR}
\begin{lstlisting}
a = 2
\end{lstlisting}
\end{LTR}
\par
مشکلی که این نوع مقداردهی دارد این است که بعد از اجرا سلولی که این کد درون آن است، این مقدار درخروجی چاپ می‌شود. برای اینکه هم مقدار موردنظر در متغیر موردنظر ذخیره شود و هم اینکه در خروحی نمایش داده نشود. دو روش وجود دارد. روش اول:
\par
\begin{LTR}
\begin{lstlisting}
a = 2;
\end{lstlisting}
\end{LTR}
\par
روش دیگری که موجود است:
\begin{LTR}
\begin{lstlisting}
a := 2
\end{lstlisting}
\end{LTR}
\par
تقاوت این دو دستور باهم این است که در روش اول مقدار حساب و ذخیره می‌شوند . ولی در روش دوم مقدار حساب نمی‌شود تا زمانی که از آن در جایی استفاده شود
\par
روش متوقف کردن برنامه در حال اجرا:
\par
\LR{Evaluation} ⇒ \LR{Quit kernel} ⇒ \LR{Local}
\par
جواب سوال پرسیده شده: Quit
\par
* هر دستور در متمتیکا باید در براکت ([ ]) نوشته شود:
\par
\begin{LTR}
\begin{lstlisting}
Sin[x] , Clear[c]
\end{lstlisting}
\end{LTR}
\par
دستورهای پایه ریاضی
\par
برای این بخش دو متغیر از قبل تعریف می‌کنیم:

\par
\begin{LTR}
\begin{lstlisting}
a = 5
b = 8
\end{lstlisting}
\end{LTR}
\par
دستور جمع:
\par
\begin{LTR}
\begin{lstlisting}
a + b        Out : 13
\end{lstlisting}
\end{LTR}
\par
دستور تفریق:
\par
\begin{LTR}
\begin{lstlisting}
a - b        Out : -3
\end{lstlisting}
\end{LTR}
\par
دستور ضرب:
\par
\begin{LTR}
\begin{lstlisting}
a * b        Out : 40
\end{lstlisting}
\end{LTR}
\par
دستور تقسیم (روش اول):
\par
\begin{LTR}
\begin{lstlisting}
a / b        Out : 5 / 8
\end{lstlisting}
\end{LTR}
\par
دستور تقسیم (روش دوم):
\par
\begin{LTR}
\begin{lstlisting}
a/b        Out : 5 / 8
\end{lstlisting}
\end{LTR}
\par
نحوه خروجی گرفتن با دقت موردنظر:
\begin{LTR}
\begin{lstlisting}
N[expression]                         e.g: N[1/3]        Out: 0.333
N[expression , accuracy]       e.g: N[1/3 , 10]        Out: 0.3333333333
\end{lstlisting}
\end{LTR}
\par
\par
برای نوشتن این روش پس از نوشتن صورت تقسیم (در این مثال a) / + Ctrl را می‌زنیم. 
\par
روش تایپ کردن فرمول‌های خاص:
\par
$Palettes \rightarrow Basic Math Assistant$
\par
استفاده از حروف یونانی و اسمبل‌ها در این قسمت نیر موجود است. 
\par
هز گزینه کلید میانبر مخصوص خود را دارد. برای آشنایی با کلید میانبر هر گزینه، روی گزینه کمی نگه دارید.
\par
* خطای 1042 : ایجاد اشکال در دستور وارد شده، مثال:
\par
\begin{LTR}
\begin{lstlisting}
W = W + 1   ,  W = W1 + 1
\end{lstlisting}
\end{LTR}
\par
روش دیگر برای نوشتن این عبارت: در این حالت کنترلی در تعداد اعشار نداریم.
\par
\begin{LTR}
\begin{lstlisting}
1 / 3 // N
\end{lstlisting}
\end{LTR}
\par
\par
\par
\par
جلسه دوم : معرفی توابع
\par
تابع رادیکال:
\par
\begin{LTR}
\begin{lstlisting}
Sqrt[4]                Out: 2
Sqrt[x**2]             Out: √(x^2)
\end{lstlisting}
\end{LTR}
\par
تابع ساده سازی (Simplify) :
\par
\begin{LTR}
\begin{lstlisting}
Simplify[%, x > 0]                         Out: +x
Simplify[%, x < 0]                         Out: -x
\end{lstlisting}
\end{LTR}
\par
در این مثال نماینده آخرین خروجی است.
\par
تابع نمایی:
\par
\begin{LTR}
\begin{lstlisting}
Exp[x]                            Out: e**x      e = 2.7
Exp[2]                            Out: e**2
Exp[2] // N                    Out: 7.38906
\end{lstlisting}
\end{LTR}
\par
تابع جزء صحیح:
\par
\begin{LTR}
\begin{lstlisting}
Floor[2.3]             Out: 2
\end{lstlisting}
\end{LTR}
\par
تابع قدرمطلق:
\par
\begin{LTR}
\begin{lstlisting}
Abs[-2]                         Out: 2
Abs[-x]                         Out: Abs[x]
\end{lstlisting}
\end{LTR}
\par
* در مثال دوم این کد، خروجی از تابع بیرون در نمی‌آید چون x می‌تواند منفی و چیزهای دیگر باشد.
\par
تابع علامت:
\par
\begin{LTR}
\begin{lstlisting}
Sign[{-2, 0, 3}]                         Out: {-1, 0, 1}
\end{lstlisting}
\end{LTR}
\par
تابع فاکتوریل (روش اول) :
\par
\begin{LTR}
\begin{lstlisting}
x = 5
Facorial[x]                       Out: 120
\end{lstlisting}
\end{LTR}
\par
تابع فاکتوریل (روش دوم) :
\par
\begin{LTR}
\begin{lstlisting}
x!                Out: 120
\end{lstlisting}
\end{LTR}
\par
تابع لگاریتم:
\par
\begin{LTR}
\begin{lstlisting}
Log[x]                Out: Log[x]
\end{lstlisting}
\end{LTR}
\par
زمانی که برای الگوریتم پایه تغریف نکنیم، نرم افزار به صورت خودکار پایه را عدد نپر ( ln x یا E ) قرار می‌دهد.
\par
\begin{LTR}
\begin{lstlisting}
Log[E]               Out: 1
\end{lstlisting}
\end{LTR}
\par
تغریف پایه الگوریتم:
\par
\begin{LTR}
\begin{lstlisting}
Log[a , x]                  a is The base of the logarithm
Log[10 , 1000]                Out: 3
\end{lstlisting}
\end{LTR}
\par
توابع مثلثاتی:
\par
\begin{LTR}
\begin{lstlisting}
Sin[x]                e,g: Sin[Pi/3]                Out: sqrt[3] / 2
Cos[x]
Tan[x]
Cot[x]
ArcSin[x]
ArcTan[x]
\end{lstlisting}
\end{LTR}
\par
به صورت دیفالت مقدار محاسبه این توابع به صورت رادیان است، اگر بخواهیم نرم افزار حاص درجه موردنظر را حساب کند، این گونه عمل می‌کنیم:
\par
\begin{LTR}
\begin{lstlisting}
Sin[30 Degree]                Out: sqrt[3] / 2
Sin[30 Degree] // N           Out: 0.5
\end{lstlisting}
\end{LTR}
\par
\par
توابع هیپربولیک
\par
\begin{LTR}
\begin{lstlisting}
Sinh[x]
Cosh[x]
Coth[x]
\end{lstlisting}
\end{LTR}
\par
تابع تولید اعداد تصادفی
\par
برای اعداد حقیقی:

\par
\begin{LTR}
\begin{lstlisting}
RandomReal[{-2, 2}, 5]     Out: {1.31496, 0.0146972, -1.486, -1.93584, 0.638742}
RandomReal[{-2, 2}, {2, 5}] Out:{{1.86401, -1.47587, 0.0531725, 1.73636, 0.243321}, 
{1.50354, 0.924516, 1.60522, 0.870081, -1.22427}}
\end{lstlisting}
\end{LTR}
\par
در مثال اول پنج عدد تصادفی در بازه {2 ,  2-} انتخاب می‌شوند.
\par
در مثال دوم دو دسته پنج تایی عدد تصادفی در بازه {2 ,  2-} انتخاب می‌شوند.
\par
برای اعداد صحیح:
\par
\begin{LTR}
\begin{lstlisting}
RandomInteger[{-2, 2}, 5]      Out: {-2, 0, 1, 2, -2}
\end{lstlisting}
\end{LTR}
\par
پنج عدد صحیح تصادفی در بازه {2 ,  2-} انتخاب می‌شوند.
\par
تبدیل عدد به عامل های اول:
\par
\begin{LTR}
\begin{lstlisting}
FactorInteger[10]     Out: {{2, 1}, {5, 1}}      10 = 2^1 * 5^1
\end{lstlisting}
\end{LTR}
\par
به توان رساندن:
\par
\begin{LTR}
\begin{lstlisting}
Superscript[2, 3]        Out: Superscript[2,3]
Superscript @@ {2, 3}     Out: Superscript[2,3]
Superscript @@@ {{2, 3}, {6, 5}}        Out: {Superscript[2,3], Superscript[6,5]}
Superscript @@ {{2, 3}, {4, 5}}         Out: Superscript[{2, 3},{4, 5}]
\end{lstlisting}
\end{LTR}
\par

\par
\begin{LTR}
\begin{lstlisting}
CenterDot[x, y]           Out: x\[CenterDot]y
CenterDot @@ (Superscript @@@ (FactorInteger[15]))  Out: Superscript[3,1]\[CenterDot]Superscript[5,1]
\end{lstlisting}
\end{LTR}
\par

کوچک‌ترین مضرب مشترک
\par
\begin{LTR}
\begin{lstlisting}
LCM[5, 6]
\end{lstlisting}
\end{LTR}
\par
بزرگترین تقسیم الیه مشترک
\par
\begin{LTR}
\begin{lstlisting}
GCD[15, 9]
\end{lstlisting}
\end{LTR}
\par
باقی‌مانده
\par
\begin{LTR}
\begin{lstlisting}
Mod[15 , 2]
\end{lstlisting}
\end{LTR}
\par
خارج قسمت
\par
\begin{LTR}
\begin{lstlisting}
Quotient[15, 3]
\end{lstlisting}
\end{LTR}
\par
ترکیب
\par
\begin{LTR}
\begin{lstlisting}
Binomial[15, 3]
\end{lstlisting}
\end{LTR}
\par
دلتار دیراک
\par
\begin{LTR}
\begin{lstlisting}
DiracDelta[x]
DiracDelta[x] // TraditionalForm   Out: \[Delta](x)
\end{lstlisting}
\end{LTR}
\par
دلتا کراینیکر
\par
\begin{LTR}
\begin{lstlisting}
DiracDelta[m, n] // TraditionalForm
\end{lstlisting}
\end{LTR}
\par
اعداد مختلط:
\par
\begin{LTR}
\begin{lstlisting}
z = x + I y    Out: x + I y
Re[z]          Out: -Im[y] + Re[x]
\end{lstlisting}
\end{LTR}
\par
برای تعریف کردن قسمت موهومی و واقعی اعداد مختلط:
\par
\begin{LTR}
\begin{lstlisting}
Refine[Re[z], Element[{x, y}, Reals]]        Out: x               // Real Part
Refine[Im[z], Element[{x, y}, Reals]]        Out: y              // Imaginary Part
\end{lstlisting}
\end{LTR}
\par
مزدوج گیری (مختلط) :
\par
\begin{LTR}
\begin{lstlisting}
Conjugate[z]        Out: Conjugate[x] - I Conjugate[y]
\end{lstlisting}
\end{LTR}
\par
ریفاین کردن

\par
\begin{LTR}
\begin{lstlisting}
Refine[Conjugate[z], Element[{x, y}, Reals]]        Out: x - I y
\end{lstlisting}
\end{LTR}
\par
تعریف تابع دلخواه:

\par
\begin{LTR}
\begin{lstlisting}
F[x_] = x^2 + 1   Out: 1 + x^2
F[3]              Out: 10
Map[F, {2, 3}]    Out: {5, 10}
F /@ {2, 3}       Out: {5, 10}
\end{lstlisting}
\end{LTR}
\par
برای تعریف توابع چند متغیره:

\par
\begin{LTR}
\begin{lstlisting}
G[x_, y_] = x + y + 2     Out: 2 + x + y
G[{2, 3}, {5, 6}]         Out: {9, 11}
\end{lstlisting}
\end{LTR}
\par
\par
\par
\par
\par
جلسه سوم : محسابات جبری و مثلثاتی، سری‌ها
بسط یک عبارت
\par
\begin{LTR}
\begin{lstlisting}
(x + 1)^3              Out: (1 + x)^3
Expand[(x + 1)^3]      Out: 1 + 3 x + 3 x^2 + x^3
(x + 1)^3 // Expand    Out: 1 + 3 x + 3 x^2 + x^3
Expand[(a + b)^2*(x + 1)^2]    Out: a^2 + 2 a b + b^2 + 2 a^2 x + 4 a b x + 2 b^2 x 
+ a^2 x^2 +  2 a b x^2 + b^2 x^2
\end{lstlisting}
\end{LTR}
\par
همانطور که مشاهده می‌شود در مثال آخر کدهای بالا هر دو جمله بسط داده می‌شوند. گاهی نیاز است که تنها یکی از این جملات بسط داده شود. برای این کار:
\par
\begin{LTR}
\begin{lstlisting}
Expand[(a + b)^2*(x + 1)^2, x]       Out: (a + b)^2 + 2 (a + b)^2 x + (a + b)^2 x^2
\end{lstlisting}
\end{LTR}
\par
همانطور که مشاهده می‌شوند فقط جمله‌ای که در آن x وجود دارد، بسط داده خواهد شد.
\par
فاکتورگیری
\par
\begin{LTR}
\begin{lstlisting}
Factor[x^4 + x^2 + x]     Out: x (1 + x + x^3)
Factor[x^4 + x^2]         Out: x^2 (1 + x^2)
\end{lstlisting}
\end{LTR}
\par
به مثال زیر توجه کنید:
\par
\begin{LTR}
\begin{lstlisting}
y = x^4 + x^2 + x
x = a + b
Factor[y]      Out: (a + b) (1 + a + a^3 + b + 3 a^2 b + 3 a b^2 + b^3)
\end{lstlisting}
\end{LTR}
\par
همانطور که مشاهده می‌شود در این کد، از (a + b) فاکتور گرفته می‌شود.
\par
ساده کردن یک عبارت کسری
\par
\begin{LTR}
\begin{lstlisting}
Cancel[(x^2 - 1)/(x - 1)]     Out: 1 + x
\end{lstlisting}
\end{LTR}
\par
تجزیه کسر به کسرهای جزئی
\par
\begin{LTR}
\begin{lstlisting}
Apart[1/((1 + x)*(5 + x))]    Out: 1/(4 (1 + x)) - 1/(4 (5 + x))
Expand[(1 + x)*(5 + x)]       Out: 5 + 6 x + x^2
Apart[1/%]                    Out: 1/(4 (1 + x)) - 1/(4 (5 + x))
\end{lstlisting}
\end{LTR}
\par
بسط عبارات مثلثاتی و تبدیل آن‌ها به عبارت نهایی
\par
\begin{LTR}
\begin{lstlisting}
Cancel[Sin[2*x]/Sin[x]]                     Out: Csc[x] Sin[2 x]
Cancel[Sin[2*x]/Sin[x], Trig  -> True]      Out: 2 Cos[x]
\end{lstlisting}
\end{LTR}
\par
ساده سازی عبارت‌های مثلثاتی
\par
\begin{LTR}
\begin{lstlisting}
TrigFactor[Cos[x + y] + Sin[x]*Sin[y]]               Out: Cos[x] Cos[y]
\end{lstlisting}
\end{LTR}
\par
بسط عبارت‌های مثلثاتی
\par
\begin{LTR}
\begin{lstlisting}
TrigExpand[Cos[x + y]]             Out: Cos[x] Cos[y] - Sin[x] Sin[y]
TrigExpand[Sin[2*x]]               Out: 2 Cos[x] Sin[x]
\end{lstlisting}
\end{LTR}
\par
تبدیل عبارت‌های مثلثاتی درجه‌های بالاتر به عبارت خطی
\par
\begin{LTR}
\begin{lstlisting}
TrigReduce[2*(Cos[x])^2]           Out: 1 + Cos[2 x]
TrigReduce[(Cos[x])^3]             Out: 1/4 (3 Cos[x] + Cos[3 x])
\end{lstlisting}
\end{LTR}
\par
تبدیل عبارت‌های مثلثاتی به نمایی و برعکس

\par
\begin{LTR}
\begin{lstlisting}
TrigToExp[Cos[x]]             Out: E^(-I x)/2 + E^(I x)/2
ExpToTrig[Exp[Ix]]            Out: Cosh[Ix] + Sinh[Ix]
\end{lstlisting}
\end{LTR}
\par
ساده سازی عبارت‌ها
\par
نکته: در نرم افزار متمتیکا در دو حالت زیر ضرب تعریف می‌شود:
\par
1) بین دو عبارت علامت ضرب " * " قرار گیرد.
\par
2) بین دو عبارت اسپیس گذاشته شود.
\par
\begin{LTR}
\begin{lstlisting}
Simplify[(x - 1) (x + 1) (x^2 + 1) + 1]            Out: x^4
Simplify[(Sin[x])^2 + (Cos[x])^2]               Out: 1
FullSimplify[Cosh[x] - Sinh[x]]               Out: E^-x
\end{lstlisting}
\end{LTR}
\par
نکته 1: FullSimplify از Simplify قوی‌تر است. کارایی هر دو یکی است.
\par
نکته 2: گاهی باید برای ساده سازی یک عبارت، ابتدا باید آن Expand شود یا در کسرها از Cancel استفاده شود.
\par
سری (مجموعه‌ها)
\par
\begin{LTR}
\begin{lstlisting}
Sum[Sin[i x], {i, 1, 5}]     Out: Sin[x] + Sin[2 x] + Sin[3 x] + Sin[4 x] + Sin[5 x]
Sum[i, {i, 1, n}]            Out: 1/2 n (1 + n)
Sum[i^2, {i, 1, n}]          Out: 1/6 n (1 + n) (1 + 2 n)
Sum[Sin[i x], {i, 1, 10, 2}] Out: Sin[x] + Sin[3 x] + Sin[5 x] + Sin[7 x] + Sin[9 x]
\end{lstlisting}
\end{LTR}
\par
روش دیگر: با استفاده از کلید میان ESC + Sum + ESC می‌توان این نماد را رسم کرد.
\begin{center}
\includegraphics[width=1\textwidth]{Sum.png}
\end{center}
\par
خروجی کدهای تصویر بالا:
\par
\begin{LTR}
\begin{lstlisting}
Out1: 1/2 n (1 + n)
Out2: 55
\end{lstlisting}
\end{LTR}
\par
حاصل ضرب:
\par
\begin{LTR}
\begin{lstlisting}
Product[Sin[i x], {i, 1, 10}]    Out: Sin[x] Sin[2 x] Sin[3 x] Sin[4 x] Sin[5 x] 
	Sin[6 x] Sin[7 x] Sin[8 x] Sin[9 x] Sin[10 x]

Product[Sin[i x], {i, 1 , 10, 2}] Out: Sin[x] Sin[3 x] Sin[5 x] Sin[7 x] Sin[9 x]
\end{lstlisting}
\end{LTR}
\par
روش دیگر: با استفاده از کلید میان ESC + Prod + ESC می‌توان این نماد را رسم کرد.
\begin{center}
\includegraphics[width=1\textwidth]{Dot.png}
\end{center}
\par
خروجی کد تصویر بالا:
\par
\begin{LTR}
\begin{lstlisting}
Out: 120
\end{lstlisting}
\end{LTR}
\par
جلسه چهارم : حد، مشتق و انتگرال
\par
بخش اول - محاسبه حد
\par
\begin{LTR}
\begin{lstlisting}
Limit[f[x], x -> x0]
Limit[f[x], x -> x0] // TraditionalForm
\end{lstlisting}
\end{LTR}
\par
خروجی کد بالا به شکل زیر می‌باشد:
\begin{center}
\includegraphics[width=1\textwidth]{limit1.png}
\end{center}
\par
مثالی دیگر:
\par
\begin{LTR}
\begin{lstlisting}
Limit[Sin[x]/x, x -> 0]    Out: 1
\end{lstlisting}
\end{LTR}
\par
نکته: اگر یک عبارت حد چپ و راست برابر نداشته باشد، حد ندارد و در نرم‌افزار پیامی طبق این مفهوم می‌آید.
\par
\begin{LTR}
\begin{lstlisting}
Limit[1/x, x -> 0]         Out: Indeterminate
\end{lstlisting}
\end{LTR}
\par
همچنین می‌توانیم حد چپ و راست را به صورت جداگانه برای یک عبارت حساب کنیم. به مثال زیر توجه کنید.
\par
حد چپ : از سمت مقادیر کمتر از صفر به سمت صفر
\par
\begin{LTR}
\begin{lstlisting}
Limit[1/x, x -> 0 , Direction -> 1]         Out: -Infinity
\end{lstlisting}
\end{LTR}
\par
حد چپ : از سمت مقادیر بیشتر از صفر به سمت صفر
\par
\begin{LTR}
\begin{lstlisting}
Limit[1/x, x -> 0, Direction -> -1]        Out: Infinity
\end{lstlisting}
\end{LTR}
\par
بخش دوم - مشتق
\par
برای انجام عمل مشتق در برنامه می‌توان به دو صورت زیر اقدام کرد:
\par
\begin{LTR}
\begin{lstlisting}
Derivative[n][f][x + 1 ]
D[f[x], {x, n}]
\end{lstlisting}
\end{LTR}
\par
که خروجی کد بالا به شکل زیر در ترمینال نمایش داده خواهد شد:
\par
\begin{center}
\includegraphics[width=1\textwidth]{div.png}
\end{center}
\par
نکته: در کد دوم پارامتر (x)f تابع موردنظر براش مشتق گرفتن، پارامتر x متغیری که می‌خواهیم طبق آن مشتق گیری انجام دهیم و پارامتر n تعداد مشتق است.
\par
به مثال زیر توجه کنید:
\par
\begin{LTR}
\begin{lstlisting}
D[(Sin[x])^10, x]                    Out: 10 Cos[x] Sin[x]^9
\end{lstlisting}
\end{LTR}
\par
در این مثال تابع (x)sin به عنوان تابع ما و x به عنوان پارامتر مشتق‌گیری قرار داده شده است. توجه شود که پارامتر n در این مثال مشخص نشده و به پیش‌فرض 1 در نظر گرفته می‌شود.
\par
مشتق جزئی:
\par
\begin{LTR}
\begin{lstlisting}
D[f[x, y], x, y]
D[f[x, y], {x, 2}, {y, 3}]
\end{lstlisting}
\end{LTR}
\par
خروجی آن در ترمینال به شکل زیر است:
\par
\begin{center}
\includegraphics[width=1\textwidth]{div1.png}
\end{center}
\par
مثال کاربردی:
\par
\begin{LTR}
\begin{lstlisting}
D[x^3 + y^2, {x, 2}, {y, 1}]               Out: 0
\end{lstlisting}
\end{LTR}
\par
بخش سوم - انتگرال گیری
\par
\begin{LTR}
\begin{lstlisting}
Integrate[f[x], x]
\end{lstlisting}
\end{LTR}
\par
که خروجی آن در ترمینال به صورت زیر نمایش داده می‌شود:
\par
\begin{center}
\includegraphics[width=1\textwidth]{int.png}
\end{center}
\par
یک مثال از انتگرال نامعین:
\par
\begin{LTR}
\begin{lstlisting}
Integrate[Sin[x], x]        Out: -Cos[x]
\end{lstlisting}
\end{LTR}
\par
انتگرال دوگانه نامعین:
\par
\begin{LTR}
\begin{lstlisting}
Integrate[f[x, y], x, y]
\end{lstlisting}
\end{LTR}
\par
که خروجی آن در ترمینال به صورت زیر نمایش داده می‌شود:
\par
\begin{center}
\includegraphics[width=1\textwidth]{int1.png}
\end{center}
\par
یک مثال از انتگرال دوگانه نامعین:
\par
\begin{LTR}
\begin{lstlisting}
Integrate[x*y + 1 , x, y]            Out: x y + (x^2 y^2)/4
\end{lstlisting}
\end{LTR}
\par
انتگرال معین:
\par
\begin{LTR}
\begin{lstlisting}
Integrate[1/(x^3 + 1), {x, 0, 1}]           Out: 1/18 (2 Sqrt[3] \[Pi] + Log[64])
\end{lstlisting}
\end{LTR}
\par
نکته: جواب انتگرال گیری‌های معین است.
\par
یک مثال از انتگرال دوگانه معین:
\par
\begin{LTR}
\begin{lstlisting}
Integrate[x*y + 1, {x, 0, t}, {y, -x, x}]          Out: t^2
\end{lstlisting}
\end{LTR}
\par
انتگرال‌های عددی:
\par
\begin{LTR}
\begin{lstlisting}
NIntegrate[Sin[Sin[x]], {x, 0, 4}]        Out: 1.45747
\end{lstlisting}
\end{LTR}
\par
اما چرا از این روش استفاده می‌شود. بیاید از روش قبل انتگرال بالا را حساب کنیم:
\par
\begin{LTR}
\begin{lstlisting}
Integrate[Sin[Sin[x]], {x, 0, 4}]
\end{lstlisting}
\end{LTR}
\par
خروجی به شکل زیر خواهد بود:

\par
\begin{center}
\includegraphics[width=1\textwidth]{int2.png}
\end{center}
\par
دلیل آن بر این است که این انتگرال به‌صورت بسته با توابع ابتدایی بیان‌شدنی نیست. همین است که Integrate در Mathematica معمولاً جواب نمادین برنمی‌گرداند.
\par
یک مثال دیگر:
\par
\begin{LTR}
\begin{lstlisting}
a = 2 ;
NIntegrate[a*Sin[Sin[x]], {x, 0, 4}]                       Out: 2.91494
\end{lstlisting}
\end{LTR}
\par
جلسه پنجم: رسم توابع
\par
5 - 1 رسم توابع در فضای دو بعد:

\par
\begin{LTR}
\begin{lstlisting}
Plot[x*Sin[1/x], {x, -1/2, 1/2}]
\end{lstlisting}
\end{LTR}
\par
خروجی:
\par
\begin{center}
\includegraphics[width=1\textwidth]{plot1.png}
\end{center}
\par
رسم سه تابع در یک دستگاه مختصات:
\par
\begin{LTR}
\begin{lstlisting}
Plot[{x*Sin[1/x], Abs[x], -Abs[x]}, {x, -1/2, 1/2}]
\end{lstlisting}
\end{LTR}
\par
خروجی:
\par
\begin{center}
\includegraphics[width=1\textwidth]{plot2.png}
\end{center}
\par
همانطور که مشاهده می‌شود هر سه تابع در یک دستگاه مختصات با رنگ‌های متفاوت رسم می‌شود. برای تغییر رنگ  و استایل توابع، در ادامه دستور قبل کد زیر نوشته می‌شود:
\par
\begin{LTR}
\begin{lstlisting}
PlotStyle -> { Red, Directive[Dashed, Blue], Yellow}
\end{lstlisting}
\end{LTR}
\par
کد کامل با استایل اضافه شده:

\par
\begin{LTR}
\begin{lstlisting}
Plot[{x*Sin[1/x], Abs[x], -Abs[x]}, {x, -1/2, 1/2}, 
 PlotStyle -> { Red, Directive[Dashed, Blue], Yellow}]
\end{lstlisting}
\end{LTR}
\par
خروجی:
\par
\begin{center}
\includegraphics[width=1\textwidth]{plot3.png}
\end{center}
\par
نکته 1 : دستور Directive برای اعمال چندین دستور برای یک نمودار استفاده می‌شود.
\par
نکته 2 : بعد از plotها نمی‌توانیم از ; استفاده کنیم، زیرا plot یک تابع است و ذخیره سازی آن ممکن نیست و صرفاً برای نمایش استفاده می‌شود.
\par
برای راهنما قرار دادن برای نمودار، بعد از تغییر رنگ در همان کروشه از دستور زیر استفاده می‌نماییم:
\par
\begin{LTR}
\begin{lstlisting}
PlotLegends -> "Expressions"
\end{lstlisting}
\end{LTR}
\par
*توضیحاتی درباره نمودار نمایش داده خواهد شد.
\par
کد کامل با بخش اضافه شده:

\par
\begin{LTR}
\begin{lstlisting}
Plot[{x*Sin[1/x], Abs[x], -Abs[x]}, {x, -1/2, 1/2}, 
 PlotStyle -> { Red, Directive[Dashed, Blue], Yellow}, 
 PlotLegends -> "Expressions"]
\end{lstlisting}
\end{LTR}
\par
خروجی:
\par
\begin{center}
\includegraphics[width=1\textwidth]{plot4.png}
\end{center}
\par
برای نمایش ندادن یکی از محور‌ها یا هر دوی آنها (محور‌های x و y) مانند زیر عمل می‌کنیم:
\par
\begin{LTR}
\begin{lstlisting}
Plot[{x*Sin[1/x], Abs[x], -Abs[x]}, {x, -1/2, 1/2}, 
 Axes -> {True, False}]
\end{lstlisting}
\end{LTR}
\par
خروجی:
\par
\begin{center}
\includegraphics[width=1\textwidth]{plot5.png}
\end{center}
\par
نکته: مقدار True به این معنی است که محور نمایش داده شود و مقدار False به این معنی است که محور نمایش داده نشود. همانطور که مشاهده می‌شود در این مثال محور x نمایش داده شده و محور y حذف شده است. اگر مقدار x هم False قرار دهیم، محور x نیز نمایش داده نمی‌شود.
\par
برای نام گذاری برای محور‌ها از دستور زیر استفاده می‌کنیم:
\par
\begin{LTR}
\begin{lstlisting}
AxesLabel -> {x, F[x]}
\end{lstlisting}
\end{LTR}
\par
کد کامل با بخش اضافه شده:

\par
\begin{LTR}
\begin{lstlisting}
Plot[{x*Sin[1/x], Abs[x], -Abs[x]}, {x, -1/2, 1/2}, 
 AxesLabel -> {x, F[x]}]
\end{lstlisting}
\end{LTR}
\par
خروجی:
\par
\begin{center}
\includegraphics[width=1\textwidth]{plot6.png}
\end{center}
\par
نکته: در این مثال، نام محور y به محور (x)F تغییر داده شده.
\par
برای جابه‌جایی نمودار از دستور زیر استفاده می‌کنیم:
\par
\begin{LTR}
\begin{lstlisting}
AxesOrigin -> {.5, .5}
\end{lstlisting}
\end{LTR}
\par
این کد به این معنی است که نمودار 0.5 واحد به سمت چپ و راست و 0.5 واحد به سمت پایین و بالا جابه‌جا شود.
کد کامل با بخش اضافه شده:

\par
\begin{LTR}
\begin{lstlisting}
Plot[{x*Sin[1/x], Abs[x], -Abs[x]}, {x, -1/2, 1/2}, 
 AxesOrigin -> {.5, .5}]
\end{lstlisting}
\end{LTR}
\par
خروجی:
\par
\begin{center}
\includegraphics[width=1\textwidth]{plot7.png}
\end{center}
\par
اگر نمودار کامل نمایش داده نشود، می‌توانیم از دستور زیر برای نمایش کامل آن استفاده کنیم:
\par
\begin{LTR}
\begin{lstlisting}
PlotRange -> Full
//or
PlotRange -> All
\end{lstlisting}
\end{LTR}
\par
کد کامل با بخش اضافه شده:
\par
\begin{LTR}
\begin{lstlisting}
Plot[{x*Sin[1/x], Abs[x], -Abs[x]}, {x, -1/2, 1/2}, PlotRange -> Full]
\end{lstlisting}
\end{LTR}
\par
خروجی:
\par
\begin{center}
\includegraphics[width=1\textwidth]{plot8.png}
\end{center}
\par
و اگر بخواهیم تا حدی مشخص نمایش داده شود، به جای Full و All آن عدد مشخص را تایپ می‌کنیم:
\par
\begin{LTR}
\begin{lstlisting}
PlotRange -> 3
\end{lstlisting}
\end{LTR}
\par
کد کامل با بخش اضافه شده:
\par
\begin{LTR}
\begin{lstlisting}
Plot[{x*Sin[1/x], Abs[x], -Abs[x]}, {x, -1/2, 1/2}, PlotRange -> 3]
\end{lstlisting}
\end{LTR}
\par
خروجی:
\par
\begin{center}
\includegraphics[width=1\textwidth]{plot9.png}
\end{center}
\par
برای استفاده از رنگ رنگین کمانی در نمودار‌های از دستور زیر استفاده می‌کنیم:
\par
\begin{LTR}
\begin{lstlisting}
ColorFunction -> Function[{x, y}, Hue[y]
\end{lstlisting}
\end{LTR}
\par
نکته: [x] Hue به این معنی است که جهت تغییر رنگ به سمت محور x است. اگر به جای x مقدار y را قرار دهیم. جهت تغییر رنگ‌ها به سمت محور y خواهد بود.
کد کامل با بخش اضافه شده:
\par
\begin{LTR}
\begin{lstlisting}
Plot[{x*Sin[1/x], Abs[x], -Abs[x]}, {x, -1/2, 1/2}, 
 ColorFunction -> Function[{x, y}, Hue[x]]]

Plot[{x*Sin[1/x], Abs[x], -Abs[x]}, {x, -1/2, 1/2}, 
 ColorFunction -> Function[{x, y}, Hue[y]]]
\end{lstlisting}
\end{LTR}
\par
خروجی:
\par
\begin{center}
\includegraphics[width=1\textwidth]{plot10.png}
\end{center}
\par
نکته: اگر از ColorFunction استفاده کنیم، نمی‌توانیم در دستور از Plot Legends نیز استفاده کنیم.
\par
5 - 2 رسم رویه‌ها
\par
سه بعدی:
\par
\begin{LTR}
\begin{lstlisting}
Plot3D[Sin[x + y^2], {x, -3, 3}, {y, -2, 2}]
\end{lstlisting}
\end{LTR}
\par
خروجی:
\par
\begin{center}
\includegraphics[width=1\textwidth]{plot11.png}
\end{center}
\par
از ColorFinder می‌توانیم در رویه‌های سه بعدی هم استفاده کنیم و بهتر است از (z)Hue در این دستور‌ها استفاده کنیم.
\par
\begin{LTR}
\begin{lstlisting}
Plot3D[Sin[x + y^2], {x, -3, 3}, {y, -2, 2} , 
 ColorFunction -> Function[{x, y, z}, Hue[z]]]
\end{lstlisting}
\end{LTR}
\par
خروجی:
\par
\begin{center}
\includegraphics[width=1\textwidth]{plot12.png}
\end{center}
\par
برای حذف چهارخانه‌های روی رویه از دستور زیر استفاده می‌کنیم:
\par
\begin{LTR}
\begin{lstlisting}
Mesh -> None
\end{lstlisting}
\end{LTR}
\par
کد کامل با بخش اضافه شده:
\par
\begin{LTR}
\begin{lstlisting}
Plot3D[Sin[x + y^2], {x, -3, 3}, {y, -2, 2}, Mesh -> None]
\end{lstlisting}
\end{LTR}
\par
خروجی:
\par
\begin{center}
\includegraphics[width=1\textwidth]{plot13.png}
\end{center}
\par
اگر به جای none از یک عدد استفاده کنیم، به تعداد آن عدد mesh در سمت x و y کشیده می‌شود:
\par
\begin{LTR}
\begin{lstlisting}
Plot3D[Sin[x + y^2], {x, -3, 3}, {y, -2, 2}, Mesh -> 4, 
 MeshStyle -> Blue]
\end{lstlisting}
\end{LTR}
\par
نکته: Blue در این کد رنگ Mesh هستش.
\par
خروجی:
\par
\begin{center}
\includegraphics[width=1\textwidth]{plot14.png}
\end{center}
\par
5 - 3 رسم توابع پارامتری
توابع دو بعدی:
\par
\begin{LTR}
\begin{lstlisting}
h = 1/(s - 1/2)^2 /. s -> Exp[I w]
ParametricPlot[{Re[h], Im[h]}, {w, 0, 2*Pi}]
\end{lstlisting}
\end{LTR}
\par
خروجی:
\par
\begin{center}
\includegraphics[width=1\textwidth]{plot15.png}
\end{center}
\par
توابع سه بعدی:
\par
\begin{LTR}
\begin{lstlisting}
ParametricPlot3D[{Sin[x], Cos[x], x/10}, {x, 0, 20}]
\end{lstlisting}
\end{LTR}
\par
خروجی:
\par
\begin{center}
\includegraphics[width=1\textwidth]{plot16.png}
\end{center}
\par
5 - 4 رسم توابع با پارامترهای نامعلوم
\par
\begin{LTR}
\begin{lstlisting}
Manipulate[Plot[a*x^2 + b*x + c, {x, -10, 10}], {a, -10, 10}, {b, -10, 
  10}, {c, -10, 10}]
\end{lstlisting}
\end{LTR}
\par
خروجی:
\par
\begin{center}
\includegraphics[width=1\textwidth]{plot17.png}
\end{center}
\par
از Manupulate نیز می‌توان برای فاکتورگیری و چندجمله‌ای ها نیز استفاده کرد:
\par
\begin{LTR}
\begin{lstlisting}
Manipulate[Factor[n^n + 1], {n, 10, 100, 13}]
\end{lstlisting}
\end{LTR}
\par
خروجی:
\par
\begin{center}
\includegraphics[width=1\textwidth]{plot18.png}
\end{center}
\par
5 - 5 رسم توابع با استفاده از نقاط
\par
\begin{LTR}
\begin{lstlisting}
ListPlot[Table[{Sin[x], Sin[2 x]}, {x, 50}]]
\end{lstlisting}
\end{LTR}
\par
خروجی:
\par
\begin{center}
\includegraphics[width=1\textwidth]{plot19.png}
\end{center}
\par
5 - 6 رسم منحنی در مختصات‌های مختلف
مختصات کروی: مثال اول
\par
\begin{LTR}
\begin{lstlisting}
SphericalPlot3D[{1, 2, 3}, {\[Theta], 0, Pi/2}, {\[CapitalPhi], 0, 
  3*Pi/2}]
\end{lstlisting}
\end{LTR}
\par
خروجی:
\par
\begin{center}
\includegraphics[width=1\textwidth]{plot20.png}
\end{center}
\par
مختصات کروی: مثال دوم
\par
\begin{LTR}
\begin{lstlisting}
PolarPlot[Sin[3*t], {t, 0, Pi}]
\end{lstlisting}
\end{LTR}
\par
خروجی:
\par
\begin{center}
\includegraphics[width=1\textwidth]{plot21.png}
\end{center}
\par
5 - 7 رسم محدوده جواب نامعدلات جبری
\par
\begin{LTR}
\begin{lstlisting}
ContourPlot[Cos[x] + Cos[y] == 1/2, {x, 0, 4*Pi}, {y, 0, 4*Pi}]
ContourPlot[{Abs[Sin[x]* Sin[y]] == .5 , 
  Abs[Cos[x]*Cos[y]] == .5}, {x, -3, 3}, {y, -3, 3}]
\end{lstlisting}
\end{LTR}
\par
خروجی:
\par
\begin{center}
\includegraphics[width=1\textwidth]{plot22.png}
\end{center}
\par
جلسه ششم: حل معادلات
6 - 1 حل معادلات جبری
\par
مثال 1 :
\par
\begin{LTR}
\begin{lstlisting}
Solve[a*x^2 + b*x + c == 0, x]         Out: {{x -> (-b - Sqrt[b^2 - 4 a c])/(2 a)}, {x -> (-b + Sqrt[b^2 - 4 a c])/(2 a)}}
\end{lstlisting}
\end{LTR}
\par
* ریشه‌های معادله درجه دو را بدست می‌آورد.
\par
مثال 2 : 
\par
\begin{LTR}
\begin{lstlisting}
Solve[x^2 + y^2 == 2 && x - y == 1 , {x, y}]         
Out: {{x -> 1/2 (1 - Sqrt[3]), 
  y -> 1/2 (-1 - Sqrt[3])}, {x -> 1/2 (1 + Sqrt[3]), 
  y -> 1/2 (-1 + Sqrt[3])}}
\end{lstlisting}
\end{LTR}
\par
* این معادله دو نقطه و دو جواب دارد که آن نقطه تقاطع این دو معادله است.
\par
برای اینکه جواب‌ها را جداگانه بدهد:
\par
\begin{LTR}
\begin{lstlisting}
sol = Solve[x^2 + y^2 == 2 && x - y == 1 , {x, y}] 
x /. sol            Out: {1/2 (1 - Sqrt[3]), 1/2 (1 + Sqrt[3])} 
\end{lstlisting}
\end{LTR}
\par
* x های جواب معدله را نشان می‌دهد.
\par
مثال 3 : 
\par
\begin{LTR}
\begin{lstlisting}
Solve[x == 1 && x == 2, x]          Out: {}
\end{lstlisting}
\end{LTR}
\par
* این دسته معادله‌ها جواب ندارند. علامت {} یعنی دستگاه جوابی ندارد.
\par
مثال 4 :
\par
\begin{LTR}
\begin{lstlisting}
Solve[x == x, x]         Out: {{}}
\end{lstlisting}
\end{LTR}
\par
* در این مثال مجمع حواب بی‌نهایت است. علامت {{}} یعنی دستگاه بی‌نهایت جواب دارد.
\par
مثال 5 : 
\par
\begin{LTR}
\begin{lstlisting}
Solve[(x^4 - 1)*(x - 4) == 0, x, Reals]         Out: {{x -> -1}, {x -> 1}, {x -> 4}}
\end{lstlisting}
\end{LTR}
\par
* ریشه‌های که فقط Real هستند در خروجی نمایش داده خواهد شد.
\par
اگر یک معادله با دستور Solve حل نشود، ممکن است با استفاده از دستور NSolve حل بشود.
\par
\begin{LTR}
\begin{lstlisting}
NSolve[(x - 1)*(x^4 - 4) == 0, Reals]        Out: {{x -> -1.41421}, {x -> 1.}, {x -> 1.41421}}
\end{lstlisting}
\end{LTR}
\par
6 - 2 یافتن ریشه معادلات غیرخطی به روش نیوتون و سکانت
\par
روش نیوتون (یک نقطه ورودی دارد)
\par
\begin{LTR}
\begin{lstlisting}
FindRoot[Sin[x] + Exp[x], {x, 0}]       Out: {x -> -0.588533}
\end{lstlisting}
\end{LTR}
\par
روش سکانت (دو نقطه ورودی دارد)
\par
\begin{LTR}
\begin{lstlisting}
FindRoot[Sin[x] + Exp[x], {x, 0, 1}]       Out: {x -> -0.588533}
\end{lstlisting}
\end{LTR}
\par
اگر بخواهیم از یک نقطه شروع کنیم و در یک بازه خاص دنبا جواب بگردد از روش زیر استفاده می‌کنیم (روش نیوتون) :
\par
\begin{LTR}
\begin{lstlisting}
FindRoot[Sin[x] + Exp[x], {x, 0, -1, 1}]      Out: {x -> -0.588533}
\end{lstlisting}
\end{LTR}
\par
* در این مثال، پارامتر دوم که در اینجا 0 قرار داده شده نقطه شروع ، دو پارامتر بعدی بازه را مشخص می‌کند. در اینجا بازه -1 تا 1.
\par
برای بالا بردن دقت محاسبات می‌توانیم از WorkingPrecision استفاده کنیم. به مثال زیر دقت کنید:
\par
\begin{LTR}
\begin{lstlisting}
FindRoot[x^2 - 2, {x, 1}]          Out: {x -> 1.41421}
FindRoot[x^2 - x, {x, 1}, WorkingPrecision -> 60]   Out: {x -> 1.00000000000000000000000000000000000000000000000000000000000}
\end{lstlisting}
\end{LTR}
\par
اگر بخواهیم دقت بسیار بسیار زیاد شود:
\par
\begin{LTR}
\begin{lstlisting}
Needs["FunctionApproximations`"]
InterpolateRoot[Exp[x] == 2 , {x, 0, 1}]   Out: {x -> 0.693147180559945309417232}
\end{lstlisting}
\end{LTR}
\par
حل نامعادله:
\par
\begin{LTR}
\begin{lstlisting}
Reduce[x^2 + y^2 < 1, {x, y}]  Out: -1 < x < 1 && -Sqrt[1 - x^2] < y < Sqrt[1 - x^2]
\end{lstlisting}
\end{LTR}
\par
رسم دایره:
\par
\begin{LTR}
\begin{lstlisting}
ContourPlot[x^2 + y^2 == 1, {x, -2, 2}, {y, -2, 2}]
\end{lstlisting}
\end{LTR}
\par
خروجی:
\par
\begin{center}
\includegraphics[width=1\textwidth]{Circle.png}
\end{center}
\par
6 - 3 معادلات دیفرانسیل معمولی و جزئی
\[
\left.
\begin{array}{l}
\text{روش تحلیلی (دقیق)} \\
\text{روش عددی (تقریبی)}
\end{array}
\right\}
\text{ حل معادله دیفرانسیل}
\]
\par
معادلاتی که می‌توان به روش تحلیلی حل کرد:
\par
\begin{LTR}
\begin{lstlisting}
DSolve[{y'[x] + y[x] == a Sin[x ]}, y, x]   Out: {{y -> Function[{x}, E^-x C[1] + 1/2 a (-Cos[x] + Sin[x])]}}
\end{lstlisting}
\end{LTR}
\par
اگر بخواهیم مقدار [1]C محاسبه شود، کد را اینگونه می‌نویسیم:
\par
\begin{LTR}
\begin{lstlisting}
DSolve[{y'[x] + y[x] == a Sin[x ], y[0] == 0}, y, x]  Out: {{y -> Function[{x}, -(1/2) a E^-x (-1 + E^x Cos[x] - E^x Sin[x])]}}
\end{lstlisting}
\end{LTR}
\par
* عبارت y[0] == 0 در کد بالا شرط اولیه است.
\par
معادلاتی که می‌توان به روش عددی (تقریبی) حل کرد:
\par
\begin{LTR}
\begin{lstlisting}
s = NDSolve[{y'[x] == y[x]*Cos[x + y[x]], y[0] == 1}, y, {x, 0, 30}]
Plot[Evaluate[y[x] /. s], {x, 0, 30}, PlotRange -> All]
\end{lstlisting}
\end{LTR}
\par
خروجی:
\par
\begin{center}
\includegraphics[width=1\textwidth]{NDSolve.png}
\end{center}
\par
حل معادلات پارشال:
\par
\begin{LTR}
\begin{lstlisting}
eq = D[\[Psi][x, t], {x, 2}] == -1/\[Alpha]^2 D[\[Psi][x, t], {t, 2}]
DSolve[eq, \[Psi][x, t], {x, t}]
\end{lstlisting}
\end{LTR}
\par
خروجی:
\par
\begin{center}
\includegraphics[width=1\textwidth]{Psi.png}
\end{center}
\par
جلسه هفتم: بردار و ماتریس
\par
نحوه ساخت یک وکتور (بردار) در متمتیکا:
\par
\begin{LTR}
\begin{lstlisting}
v = {1, 2, 3}         Out: {1, 2, 3}
\end{lstlisting}
\end{LTR}
\par
نحوه ساخت یک ماتریس:
\par
\begin{LTR}
\begin{lstlisting}
y = {{1, 2}, {3, 4}}    Out: {{1, 2}, {3, 4}}      // 2×2 matrix
\end{lstlisting}
\end{LTR}
\par
اگر بخواهیم به شکل ماتریس نمایش داده شود:
\par
\begin{LTR}
\begin{lstlisting}
y = {{1, 2}, {3, 4}} // MatrixForm
\end{lstlisting}
\end{LTR}
\par
خروجی:
\par
\begin{center}
\includegraphics[width=1\textwidth]{Matrix1.png}
\end{center}
\par
روش ساخت یک بردار و ماتریش با دستور Table :
\par
\begin{LTR}
\begin{lstlisting}
Table[i^2, {i, 5}]    Out: {1, 4, 9, 16, 25}
\end{lstlisting}
\end{LTR}
\par
اگر ماتریش شما از قانون خاصی پیروی نمی‌کند( یک ماتریس خاص است) باید خود ماتریس در []Table نوشت.
\par
مثالی دیگر:
\par
\begin{LTR}
\begin{lstlisting}
Table[i^2, {i, 5, 10}]            Out: {25, 36, 49, 64, 81, 100}
Table[i^2, {i, 5, 10, 2}]      Out: {25, 49, 81}
\end{lstlisting}
\end{LTR}
\par
\par
\begin{LTR}
\begin{lstlisting}
Table[i^2, {i, 5}, {j, 5}] // MatrixForm
Table[j^3, {i, 5}, {j, 5}] // MatrixForm
\end{lstlisting}
\end{LTR}
\par
خروجی:
\par
\begin{center}
\includegraphics[width=1\textwidth]{Matrix2.png}
\end{center}
\par
* نجوه و ترتیب قرارگیری i و j در Table بسیار مهم است.
\par
7 - 1 اعمال اصلی و گرفتن درایه از ماتریس
\par
\begin{LTR}
\begin{lstlisting}
M = Table[j^2, {i, 5}, {j, 5}] // MatrixForm
M + M // MatrixForm;               // Summation
M . M // MatrixForm;               // Multipication
\end{lstlisting}
\end{LTR}
\par
در مثال زیر درایه واقع در سطر و ستون اول ماتریس M را مقدار 6 قرار می‌دهیم:
\par
\begin{LTR}
\begin{lstlisting}
M[[1, 1]] = 6
M[[1, 1]]               Out: 6
\end{lstlisting}
\end{LTR}
\par
7 - 2 ضرب داخلی و خارجی و محاسبه نورم اول بردار
\par
ضرب داخلی (یک عدد است):
\par
\begin{LTR}
\begin{lstlisting}
v = {1, 2, 3}             
v . v            Out: 14
Dot[v, v]     Out: 14
\end{lstlisting}
\end{LTR}
\par
ضرب خارجی (یک بردار است) : 
\par
\begin{LTR}
\begin{lstlisting}
Cross[v, v]       Out: {0, 0, 0}
v * v         Out: {0, 0, 0}
\end{lstlisting}
\end{LTR}
\par
بدست آوردن اندازه نورم اول بردار:
\par
\begin{LTR}
\begin{lstlisting}
Norm[v]       Out: Sqrt[14]
\end{lstlisting}
\end{LTR}
\par
7 - 3 محاسبه دترمینان، جمغ درایه‌های قطر اصلی (تریس ماتریس)، معکوس ماتریس، تولید ماتریس‌های خاص
محاسبه دترمینان ماتریس - مثال 1 :
\par
\begin{LTR}
\begin{lstlisting}
M = Table[j^2, {i, 5}, {j, 5}] ;
Det[M]    Out: 0        
\end{lstlisting}
\end{LTR}
\par
محاسبه دترمینان ماتریس - مثال 2 :
\par
\begin{LTR}
\begin{lstlisting}
m = {{1, 2}, {3, 4}}
Det[m]         Out: -2
\end{lstlisting}
\end{LTR}
\par
محاسبه تریس ماتریس:
\par
\begin{LTR}
\begin{lstlisting}
Tr[m]       Out: 5
\end{lstlisting}
\end{LTR}
\par
معکوس ماتریس - مثال 1 :
\par
\begin{LTR}
\begin{lstlisting}
Inverse[m]       Out: {{-2, 1}, {3/2, -(1/2)}}
\end{lstlisting}
\end{LTR}
\par
معکوس ماتریس - مثال 2 ‌:
\par
\begin{LTR}
\begin{lstlisting}
Inverse[{{1, 2}, {1, 2}}]      Out: Inverse::sing: Matrix {{1,2},{1,2}} is singular.
\end{lstlisting}
\end{LTR}
\par
* در این مثال چون ماتریس singular هست، برنامه ارور می‌دهد. این ماتریس معکوس نیست زیرا که دترمینان آن مساوی با صفر است.
\par
محاسبه رنک یک ماتریس:
\par
\begin{LTR}
\begin{lstlisting}
MatrixRank[{{1, 2}, {1, 2}}]      Out: 1
\end{lstlisting}
\end{LTR}
\par
* اگر رنک یک ماتریکس n باشد کامل است و معکوس‌پذیر و در غیر این صورت معکوس‌ناپذیر است.
\par
تشکیل ماتریس‌های خاص
\par
ماتریس همانی
\par
\begin{LTR}
\begin{lstlisting}
IdentityMatrix[3] // MatrixForm
\end{lstlisting}
\end{LTR}
\par
خروجی:
\par
\begin{center}
\includegraphics[width=1\textwidth]{Matrix3.png}
\end{center}
\par
ماتریس قطری (همانی نباشد) :

\par
\begin{LTR}
\begin{lstlisting}
DiagonalMatrix[{1, 2, 3}] // MatrixForm
\end{lstlisting}
\end{LTR}
\par
خروجی:
\par
\begin{center}
\includegraphics[width=1\textwidth]{Matrix4.png}
\end{center}
\par
7 - 4 حل دستگاه خطی همگن و ناهمگن
\par
\begin{LTR}
\begin{lstlisting}
LinearSolve[{{1, 1}, {1, -1}}, {0, 0}]          Out: {0, 0}
LinearSolve[{{1, 1}, {1, -1}}, {1, 0}]          Out: {1/2, 1/2}
\end{lstlisting}
\end{LTR}
\par
مقادیر و بردار‌های ویژه یک ماتریس:
\par
\begin{LTR}
\begin{lstlisting}
M = {{1, 2}, {3, 4}};
Eigenvalues[M]         Out: {1/2 (5 + Sqrt[33]), 1/2 (5 - Sqrt[33])}
Eigenvectors[M]          Out: {{1/6 (-3 + Sqrt[33]), 1}, {1/6 (-3 - Sqrt[33]), 1}}
\end{lstlisting}
\end{LTR}
\par
\end{document}